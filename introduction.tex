%%%%%%%%%%%%%%%%%%%%%%%%%%%%%%%%%%%%%%%%%%%%%%%%%%%%%%%%%%%%%%%%%%%%%%%%%%%%%%%%
\intro
%%%%%%%%%%%%%%%%%%%%%%%%%%%%%%%%%%%%%%%%%%%%%%%%%%%%%%%%%%%%%%%%%%%%%%%%%%%%%%%%

Написание безопасных и надежных программ является сложной задачей для программиста,
некоторые ошибки можно легко найти на стадии компиляции, но не все.
Так, обращение к еще не инициализированным полям является частой причиной возникновения ошибок
в объектно-ориентированных языках программирования.
Такие ошибки тяжело находить, а поведение программы может становиться не предсказуемым.

Целью данной работы является разработка анализа для нахождения ошибок инициализации для языка Kotlin.
Также в этой работе создается прототип системы безопасной инициализации, и производится оценка его производительности.

Работа состоит из 6 разделов.
Первый раздел посвящен проблеме.
Включает в себя ряд примеров не безопасной инициализации и ряд обязательных требований без которых анализ будет не состоятельным.

Второй раздел включает в себя обзор существующих решений в индустриальных языках и теоретических подходов
к решению проблемы безопасной инициализации.
Также в данном разделе выделяются достоинства и недостатки каждого из подходов.

Третий раздел включает в себя постановку задачи.
\mytodo{И что-то еще что должен включать этот раздел}

Четвертый раздел посвящен дизайну анализа инициализированности.
В данном разделе выбирается подход на основе, которого будет создаваться безопасная инициализация для языка Kotlin.
Также рассмотрены проблемы, которые пришлось решать, чтобы выбранный подход мог работать для языка Kotlin.

В пятом разделе рассказывается про детали реализации прототипа и объясняются основные идеи анализа.

Шестой раздел включает в себя тестирование написанного прототипа, а также оценку производительности и качества анализа.
Рассказывается на каких проектах был запущен прототип и какие результаты он показал.
