%%%%%%%%%%%%%%%%%%%%%%%%%%%%%%%%%%%%%%%%%%%%%%%%%%%%%%%%%%%%%%%%%%%%%%%%%%%%%%%%
%%
%% Настройка параметров документа
%%
%%%%%%%%%%%%%%%%%%%%%%%%%%%%%%%%%%%%%%%%%%%%%%%%%%%%%%%%%%%%%%%%%%%%%%%%%%%%%%%%

% Быть посвободнее при склеивании слов
\sloppy

% Настройка листингов
\renewcommand{\lstlistingname}{Листинг}
\lstset{
	frame=single, % adds a frame around the code
	rulesepcolor=\color{gray},
	rulecolor=\color{black},
	breaklines=true,
	xleftmargin=2em,
	extendedchars={true},
	inputencoding={utf8},
	basicstyle={\ttfamily \scriptsize},
	keywordstyle={\rmfamily \bfseries},
	commentstyle={\rmfamily \itshape},
	tabsize={2},
	numbers={left},
	frame={single},
	showstringspaces={false},
}
\lstdefinestyle{java}{
	breaklines={true},
	texcl=true,
	language={Java},
}
\input{listings_cyr_hack}

% Настройка стиля оглавления
% \renewcommand{\tocchapterfont}{}

% Релизная версия 
\newcommand{\isRelease}{}
% \newcommand{\isRelease}{disable}

% Настройка todo
\newcommand{\mytodo}[1]{\todo[linecolor=orange,backgroundcolor=orange!25,bordercolor=orange,\isRelease,inline]{TODO #1}}
\newcommand{\unsure}[2][1=]{\todo[linecolor=red,backgroundcolor=red!25,bordercolor=red,\isRelease,#1]{#2}}
\newcommand{\change}[2][1=]{\todo[linecolor=blue,backgroundcolor=blue!25,bordercolor=blue,\isRelease,#1]{#2}}
\newcommand{\info}[2][1=]{\todo[linecolor=OliveGreen,backgroundcolor=OliveGreen!25,bordercolor=OliveGreen,\isRelease,#1]{#2}}
\newcommand{\improvement}[2][1=]{\todo[linecolor=Plum,backgroundcolor=Plum!25,bordercolor=Plum,\isRelease,#1]{#2}}
\newcommand{\shadowtodo}[1]{\todo[disable]{#1}}
%

% Настройка autoref
\renewcommand{\figureautorefname}{рис.}
\renewcommand{\tableautorefname}{табл.}
%

%%%%%%%%%%%%%%%%%%%%%%%%%%%%%%%%%%%%%%%%%%%%%%%%%%%%%%%%%%%%%%%%%%%%%%%%%%%%%%%%
