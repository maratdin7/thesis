\keywords{%
    Безопасная инициализация,
    Kotlin,
    Система типов и эффектов
}

\abstractcontent{
    Тема выпускной квалификационной работы:  <<Анализ инициализированности в языке Kotlin>>.

    Данная работа посвящена созданию анализа для нахождения ошибок инициализации в языке Kotlin
    на основе подхода представленного для Scala 3\cite{safe-initialization-for-scala}
    \unsure {Должна ли она тут быть и нужно лу вообще упоминать скалу}.
    В ходе данной работы решались следующие задачи:
    \begin{itemize}
        \item{}Анализ существующих решений по поиску ошибок инициализации в языках.
        \item Разработка анализа для нахождения не безопасной инициализации в языке Kotlin.
        \item Создание прототипа, реализующего данный подход.
        \item Оценка производительности и тестирование данного подхода.
    \end{itemize}

    Анализ основывается на подходе, который был представлен для языка Scala 3\cite{safe-initialization-for-scala}
    и модифицирует его для работы в языке Kotlin. Анализ основывается на системе типов и эффектов, и способен находить
    ряд проблем с инициализацией, которые существующий анализ находить не способен.
}

\keywordsen{
    safe initialization,
    Kotlin,
    Type-and-effects system
}

\abstractcontenten{
    \mytodo{Hello}
}
